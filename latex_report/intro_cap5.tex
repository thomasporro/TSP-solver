In this section, the models described in this report are compared with the performance profile tool provided. Several runs were carried out to fully test the models for a more accurate conclusion. In particular, the best algorithm is the one that finds the optimal solution in the shortest time for most of the time. While analyzing the metaheuristics approaches it is impossible to compare them in a time scale, so instead of the time the comparison here is done between the cost of the solution obtained.

Since the TSP is a NP-hard problem, finding a winner is not always easy. Sometimes different methods can lead to very similar results, so several factors - such as consistency - are took in consideration.

To discuss, the outcome of the tests is plotted and so it is needed some explanation on how to understand them. While the x-axis portrays the time (or cost) ratio which will show what algorithm performs better, the y-axis shows the fraction of instances that the algorithm ends up winning. In general, the best method is the one that remains more on the left side of the chart.

As explained in section \ref{sec:solution-data-management}, the TSPLIB is used only to carry out if the implemented algorithms can reach the optimal value. To run these tests, a series of randomly generated instances are used.

\section{Compact models}
\label{sec:results-compact}
In this section the compact models presented in chapter \ref{chapter:compact-models} are evaluated focusing in particular on MTZ and its variants and and GG models. To test them, a set of 20 instance of 50 nodes was generated and  was runned with a global timelimit of 30 minutes. The results are visible in figure \ref{fig:result-compact}.

\begin{figure}[h]
	\centering
	\includegraphics[width=0.6\textwidth]{images/final_mtz_mtzlazy_gg.png}
	\caption{The comparison chart of the compact models.}
	\label{fig:result-compact}
\end{figure}

From this image, it can be stated that the MTZ basic model is far from being the best one, and the best models are GG and MTZ\_LAZY - which perform in a comparable way. However, it is possible to see the that GG model deliver a consistent solution of the instance, while MTZ\_LAZY has reached the timelimit in some cases.

\section{SEC methods}
\label{sec:results-sec}
To test the algorithms presented in section \ref{chapter:other-sec}, a set of 20 instances with 350 nodes was created.

\begin{figure}[h]
	\centering
	\includegraphics[width=0.6\textwidth]{images/final_SEC.png}
	\caption{The comparison chart of the SEC methods.}
	\label{fig:result-sec}
\end{figure}

The results in figure \ref{fig:result-sec} showed clearly that the Branch and Cut method is the best approach of this comparison. The most outstanding outcome is that the B\&C (Branch-and-Cut) relaxation method was in a trailing position even compared with the Benders method, while I was assuming that its performance was comparable with the normal B\&C.

The meaning of this is the fact that the callback is applied each time a fractional solution is found. This suggests that it is called way more times than in the previous one and this leads to worse performance. For solving this problem, I have done deeper research on this approach. Figure \ref{fig:result-bac} shows the B\&C relaxation applied with a different probability.

\begin{figure}[h]
	\centering
	\includegraphics[width=0.6\textwidth]{images/branch_perc.png}
	\caption{The comparison between different B\&C relaxation.}
	\label{fig:result-bac}
\end{figure}

This plot shows that the best performing method is the one with a percentage of 40\%. In this way is possible to obtain the final chart with the best methods.

\begin{figure}[h]
	\centering
	\includegraphics[width=0.6\textwidth]{images/final_final_SEC.png}
	\caption{Final comparison between different B\&C.}
	\label{fig:result-final-bac}
\end{figure}

This last chart shows that the classic Branch and Cut method and the relaxation one applied with a percentage of 40\% are similar, but the one that solves only the integer solution is considered the best among them.



\section{Matheuristics}
\label{sec:results-math}
The algorithms from section \ref{chapter:metaheuristics} are analyzed in this segment. As explained in their relative chapter, for analyzing them properly a starting model has to be applied to whole the system. Having seen the previous results, the best option is choosing the classic Branch-and-Cut method since it is the one that can provide a faster solution to the instances.

To test the algorithms - which are the Hard-Fixing and Soft-Fixing - a set of 20 instances with 1000 nodes is randomly generated, with 30 minutes as time limit. It is essential to remark that for each method the number of edges blocked is variable, and each time a better solution is found this total number decreases. The results are visible in figure \ref{fig:result-math}.

\begin{figure}[h]
	\centering
	\includegraphics[width=0.6\textwidth]{images/final_math.png}
	\caption{The comparison between the Matheuristics.}
	\label{fig:result-math}
\end{figure}

It is possible to state that - except for particular instances - these two methods are providing the same performances. On the other hand, if we zoom the left side of the previous chart we can see the real difference between the algorithms.

\begin{figure}[h]
	\centering
	\includegraphics[width=0.6\textwidth]{images/final_math_zoom.png}
	\caption{The comparison between the Matheuristics zoomed.}
	\label{fig:result-math-zoom}
\end{figure}

In figure \ref{fig:result-math-zoom} it is indisputable that the Soft-Fixing approach, with the exclusion of the particular worse results, is the best one of this section.

\section{Metaheuristics}
\label{sec:results-heur}
This is the last comparison between the methods presented in this project. In this section, metaheuristics presented in chapter \ref{chapter:metaheuristics} are compared: the VNS, the Tabu Search, and the Genetic algorithms.

To compare them, larger instances than the previous ones are created. A set of 20 instances with 2000 nodes is randomly built and then they are tested with a time limit of 30 minutes. The results are visible in figure \ref{fig:result-meta}.

\begin{figure}[h]
	\centering
	\includegraphics[width=0.6\textwidth]{images/final_meta.png}
	\caption{The comparison between the Matheuristics.}
	\label{fig:result-meta}
\end{figure}

It is possible to notice that genetic algorithms lead to worst solutions. This is due to the fact that the starting population is totally randomly generated and so, within a small time limit, it is difficult to improve as much as in VNS or Tabu Search.

To see the winner of this section it is necessary to zoom the left side of the chart. 

\begin{figure}[h]
	\centering
	\includegraphics[width=0.6\textwidth]{images/final_meta_zoom.png}
	\caption{The comparison between the Matheuristics.}
	\label{fig:result-meta-zoom}
\end{figure}

In figure \ref{fig:result-meta-zoom} it is evident that the winner of this comparison is the VNS.

\section{Conclusions}
\label{sec:conclusions}
Among all the exact algorithms there is no doubt that the best performing one is the B\&C method, which reached the optimal solution in the shortest amount of time. Given a sufficient time limit, it should be able to solve larger instances with enough efficiency.

The use of a compact model is infeasible since the effort to compute an optimal solution is enormous and surely not worth it. However in this category, the best performing one is the GG model because it delivers consistently an optimal solution.

By all the matheuristic approaches the best-performing one was the Soft-Fixing, which brings the best solution most of the time. The Hard-Fixing method is still usable though because the solutions obtained from it are not too far away from the best ones.

The best metaheuristic approach is the VNS, which always provides the better solution. The use of the Soft-Fixing method is possible even with a higher number of nodes but - given a short amount of time - the solver hardly reaches a feasible solution.